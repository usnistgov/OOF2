\documentclass[10pt]{article}
\pdfcompresslevel = 9
\pdfoutput = 1
\usepackage[pdftex]{graphicx}
% \renewcommand{\thesection}{\arabic{section}}
% \renewcommand{\thesubsection}{\arabic{section}.\arabic{subsection}}
\begin{document}
\title{Ramberg-Osgood Elasticity Property for OOF}
\author{Andrew Reid}
\date{Dec. 16, 2008}
\maketitle

This software package provides a particular form of nonlinear
elasticity property for the OOF software.  It is written as an OOF
extension, and so needs to be separately built and installed in order
to run within OOF.  It is compatible with the 2.1 version of the
OOF code base.

\section{Constitutive Rule}

Ramberg-Osgood elasticity is a form of nonlinear elasticity designed
to emulate plastic yield.  Unlike true plasticity, there is no
stress-free strain in the Ramberg-Osgood scheme --- deformation
following this rule is fully recovered when the load is removed.

The strain as a function of the stress is given by

\begin{equation}
\varepsilon_{ij}={1+\nu \over E}\sigma_{ij}-
	  {\nu \over E}\sigma_{kk}\delta_{ij}+
          {{3 \over 2} \alpha}\left({q \over s_0}\right)^{n-1}
	    (\sigma_{ij}-{1 \over 3}\sigma_{kk}\delta_{ij})
\label{eqn:base} 
\end{equation}
where $\varepsilon_{ij}$ is the strain tensor, $\sigma_{ij}$ is the
stress tensor, $E$ and $\nu$ are the Young's modulus and Poisson ratio
of the material, $s_0$ is a reference stress value, which may be
usefully thought of as the yield stress.  $\alpha$ is a constitutive
parameter with dimensions of inverse stress, and $q$ is the Von Mises
equivalent stress, given by
\[
q=\sqrt{{3 \over 2}(\sigma_{ij}\sigma_{ij}-{1 \over 3}\sigma_{kk}^2)}
\]

This can also be expressed slightly more awkwardly in terms of the
tensor elastic constants, which are more directly available in OOF:

\begin{equation}
\varepsilon_{ij}=A\sigma_{ij}-
	  B\sigma_{kk}\delta_{ij}+ 
          {{3 \over 2} \alpha}\left({q \over s_0}\right)^{n-1}
	  (\sigma_{ij}-{1 \over 3}\sigma_{kk}\delta_{ij})
\label{eqn:rule}
\end{equation}

with

\begin{equation}
A={1 \over c_{11}-c_{12}} \qquad\mathrm{and}\qquad 
B={c_{12} \over (c_{11}-c_{12})(c_{11}+2c_{12})}.
\end{equation}

The first two terms of the constitutive rule, Eq.~\ref{eqn:rule}, are
those of conventional linear elasticity.  The third term provides for
potentially rapid increases in strain for modest increases in the
shear part of the stress, which is characteristic of a yielding
material.

\section{The OOF Property}

This OOF property uses the constitutive rule above, and provides for
user input of the parameters $\alpha$, $n$, and $s_0$, along with the
isotropic linear elastic parameters in whatever form the user prefers.

The architecture of the OOF software is such that properties are asked
to specify their fluxes in terms of the fields being solved for in the
computation. For elasticity, this means that constitutive rules should
give the stress as a function of the strain.

Obviously, the constitutive rule in Eq.~\ref{eqn:rule} does not do
this.  In the present implementation, what the property does is, for
each evaluation point, first determine the local value of the strain,
and then numerically compute the local stress by solving
Eq.~\ref{eqn:rule} via a Newton-Raphson scheme.  Once the local stress
is known, the property can then compute the local compliance matrix,
which is the derivative of Eq.~\ref{eqn:rule} with respect to stress.
The local stiffness matrix is then the inverse of this matrix.  An
additional stress offset is also computed, so that what is finally
returned to the OOF software is the best local linear approximation to
the constitutive rule, with the independent variable being the strain.

The OOF code then takes care of solving this problem using an
appropriate nonlinear solver, iterating the system as a whole to
convergence.

\section{Building}

First, make sure that OOF2 is built and installed.

The extension source code and a setup.py script are located in the
ramberg\-osgood directory of the OOFEXTENSIONS package.  This code
requires SWIG version 1.1 build 883 in order to work. Unlike the OOF
distribution, it is not possible to skip running SWIG when building
this extension.  If a functional OOF2 installation is present on the
system, and SWIG is available, it should be sufficient to cd to the
directory of the setup script, and run:
\begin{obeylines}
\tt{
> python setup.py install --prefix=<oof-installation-target>
}
\end{obeylines}
The installation target should be the same one that was used when
installing the main OOF program. If this target is not specified, it
defaults to /usr/local.

Once installed, the property will not automatically be loaded into the
OOF software. To accomplish this, either start OOF2 with the
{\tt--import} command line argument, like this:
\begin{obeylines}
\tt
  oof2 --import ramberg\_osgood
\end{obeylines}
or after it's started, open an OOF console (or run OOF in
text mode) and type:
\begin{obeylines}
\tt{
>>> import ramberg\_osgood
}
\end{obeylines}
The Ramberg-Osgood elasticity property should then appear in the list
of available properties under Mechanical and Elasticity in the
property pane on the OOF materials page.

\section{Disclaimer}

This software was produced by NIST, an agency of the United States
government, and by statute is not subject to copyright in the United
States. Recipients of this software assume all responsibilities
associated with its operation, modification, and maintenance. However,
to facilitate maintenance, we ask that before distributing modified
versions of this software, you first contact the authors at
oof\_manager@nist.gov.

\end{document}