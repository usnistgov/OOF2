\documentclass[10pt,a4paper]{article}

\bibliographystyle{plain}
\usepackage{latexsym}
\usepackage{algorithm,algorithmic}
\usepackage{amssymb,amsthm}
\usepackage{amsmath,amsfonts}
\usepackage{mathrsfs}
\usepackage{enumerate}
\usepackage{graphicx,psfrag}
\usepackage{comment}
%\usepackage[notcite,notref]{showkeys}

\newtheorem{theorem}{Theorem}[section]
\newtheorem{remark}{Remark}[section]
\newtheorem{lemma}{Lemma}[section]
\newtheorem{corollary}{Corollary}[section]
\newtheorem{assumption}{Assumption}[section]
\newtheorem{definition}{{Definition}}[section]
\newtheorem{proposition}{Proposition}[section]
\newtheorem{question}{Question}


\author{}
\title{Finite Element Discretization of \\ the Static Linear PDE}

\begin{document}

\maketitle

We derive the finite element (FE) discretization of the following equations:
%
\begin{equation}\label{E:PDE}
\left\{
\begin{array}{rcl}
-\mathrm{div}(k(x)\nabla u(x)) + f(x) &=& 0 \ \ \mathrm{in}\ \Omega, \\
u(x) &=& g(x) \ \mathrm{on} \ \ \Gamma_D \quad (\mathrm{Dirichet \ B.C.}), \\
k(x) \frac{\partial u}{\partial n}(x) &=& h(x) 
\ \mathrm{on} \ \ \Gamma_N \quad (\mathrm{Neumann \ B.C.}), \\
u(x) &=& c+p(x) \ \ \mathrm{on} \ \Gamma_F \quad (\mathrm{Floating \ B.C.}),
\end{array}
\right.
\end{equation}
%
where $u(x), c$ are unknown; 
$f(x), g(x), h(x), p(x)$ are given known
functions and the boundary of $\Omega$ is 
$\partial\Omega = \Gamma_D \cup \Gamma_N \cup \Gamma_F$.\\
The floating boundary condition implies 
%
\begin{equation}\label{E:balance-eqn}
\int_{\Gamma_F} \sigma\cdot n 
= \int_{\Gamma_F} k\frac{\partial u}{\partial n} = 0.
\end{equation}
%
We write the weak form of \eqref{E:PDE} by multiplying 
the first equation with a test function $\phi$.
%
\begin{align*}
-&\int_\Omega \phi \mathrm{div}(k\nabla u) + \int_\Omega f\phi = 0, \\
&\int_\Omega k\nabla\phi \nabla u + \int_{\partial\Omega} k\frac{\partial u}{\partial n} \phi +\int_\Omega f\phi = 0, \\
&\int_\Omega k\nabla\phi \nabla u + \int_{\Gamma_D} k\frac{\partial u}{\partial n} \phi + \int_{\Gamma_N} k\frac{\partial u}{\partial n} \phi + \int_{\Gamma_F} k\frac{\partial u}{\partial n} \phi +\int_\Omega f\phi = 0, \\
&\int_\Omega k\nabla\phi \nabla u + \int_{\Gamma_D} k\frac{\partial u}{\partial n} \phi  + \int_{\Gamma_F} k\frac{\partial u}{\partial n} \phi +\int_\Omega f\phi 
+ \int_{\Gamma_N} h \phi = 0.
\end{align*}
%
We replace $u$ with a function $v$ satisfying $u = v + u_0$, 
where $u_0$ is a function such that $u_0(x) = g(x) \ \mathrm{on} \ \Gamma_D$
and $u_0(x) = p(x) \ \mathrm{on} \ \Gamma_F$. 
We obtain the following equation
%
\begin{equation*}
\int_\Omega k\nabla\phi \nabla v + \int_{\Gamma_D} k\frac{\partial u}{\partial n} \phi + \int_{\Gamma_F} k\frac{\partial u}{\partial n} \phi + \int_\Omega f\phi + \int_{\Gamma_N} h\phi + \int_\Omega k\nabla\phi \nabla u_0= 0,
\end{equation*}
%
where the primary unknown is $v$ satisfying 
$v(x) = 0 \ \mathrm{on} \ \Gamma_D$ and 
$v(x) = c \ \mathrm{on} \ \Gamma_F$.


To define a well-posed variational problem (and to formulate
the FE discretization based on that), we choose two versions
$\varphi$ and $\psi$ (both in $H^1(\Omega)$) of the test 
functions $\phi$. The functions $\varphi$ satisfy $\varphi=0$ 
on $\Gamma_D \cup \Gamma_F$. The functions $\psi$ satisfy
$\psi=1$ on $\Gamma_F$, $\psi=0$ on $\Gamma_D$.
This results in two equations:
%
\begin{align}
\int_\Omega k\nabla\varphi \nabla v 
+ \underbrace{\int_{\Gamma_D} k\frac{\partial u}{\partial n} \varphi}_{\begin{array}{c} = 0 \\ (\varphi=0 \ \mathrm{on} \ \Gamma_D) \end{array}} 
+ \underbrace{\int_{\Gamma_F} k\frac{\partial u}{\partial n} \varphi}_{\begin{array}{c} = 0 \\ (\varphi=0 \ \mathrm{on} \ \Gamma_F) \end{array}}
+ \int_\Omega f\phi + \int_{\Gamma_N} h\varphi 
+ \int_\Omega k\nabla\varphi \nabla u_0 &= 0, \nonumber
\\
\int_\Omega k\nabla\psi \nabla v 
+ \underbrace{\int_{\Gamma_D} k\frac{\partial u}{\partial n} \psi}_{\begin{array}{c} = 0 \\ (\psi=0 \ \mathrm{on} \ \Gamma_D) \end{array}}
+ \underbrace{\int_{\Gamma_F} k\frac{\partial u}{\partial n} \psi}_{\begin{array}{c}= \int_{\Gamma_F} k\frac{\partial u}{\partial n} = 0 \\ 
(\mathrm{by \ eqn \ \eqref{E:balance-eqn}}) \end{array}}
+ \int_\Omega f\psi + \int_{\Gamma_N} h\psi 
+ \int_\Omega k\nabla\psi \nabla u_0 &= 0, \nonumber
\end{align}
%
or
\begin{equation}\label{E:weak-form}
\begin{aligned}
\int_\Omega k\nabla\varphi \nabla v + \int_\Omega f\varphi 
+ \int_{\Gamma_N} h\varphi + \int_\Omega k\nabla\varphi \nabla u_0 &=& 0, \\
\int_\Omega k\nabla\psi \nabla v + \int_\Omega f\psi + \int_{\Gamma_N} h\psi 
+ \int_\Omega k\nabla\psi \nabla u_0 &=& 0.
\end{aligned}
\end{equation}
%
To obtain the FE discretization, we choose a basis set 
$\{\varphi_i\}^n_{i=1} \cup \{\Psi\}$ such that $\varphi=0$ 
on $\Gamma_D \cup \Gamma_F$ and $\Psi=0$ on $\Gamma_D$,
$\Psi=1$ on $\Gamma_F$. The basis function $\Psi$ can be
defined by $\Psi = \sum^{n_F}_{l=1} \psi_l$ where 
$\{\psi_l\}^{n_F}_{l=1}$ are the nodal basis functions
on $\Gamma_F$. The FE solution for $v$ is given by
\[
v(x) \simeq v_j \varphi_j(x) + c \Psi(x).
\]
We use this and the interpolant 
$u_0 \simeq g_j \varphi_j(x) + p_l \psi_l(x)$
to write the discretized version of the integral equations
\eqref{E:weak-form}.
%
\begin{align*}
&v_j \int_\Omega k\nabla\varphi_i \nabla\varphi_j 
+ c \int_\Omega k\nabla\varphi_i \nabla\Psi
+ \int_\Omega f\varphi_i + \int_{\Gamma_N} h\varphi_i 
+ g_j \int_\Omega k\nabla\varphi_i \nabla\phi_j
+ p_l \int_\Omega k\nabla\varphi_i \nabla\psi_l = 0, 
\\
&v_j \int_\Omega k\nabla\Psi \nabla\varphi_j 
+ c \int_\Omega k\nabla\Psi \nabla\Psi
+ \int_\Omega f\Psi + \int_{\Gamma_N} h\Psi 
+ g_j \int_\Omega k\nabla\Psi \nabla\phi_j
+ p_l \int_\Omega k\nabla\Psi \nabla\psi_l = 0.
\end{align*}
%
To write the corresponding linear system, we define the following 
matrices and vectors
%
\begin{equation*}
\begin{gathered}
K_{ij} := \int_\Omega k\nabla\varphi_i\nabla\varphi_j, \qquad
K^F_{il} := \int_\Omega k\nabla\varphi_i\nabla\psi_l, \\
\widetilde{K}_{i} := \int_\Omega k\nabla\varphi_i\nabla\Psi
= \sum^{n_F}_{l=1} K_{il} , \qquad
\widetilde{K}^F_{l} := \int_\Omega k\nabla\Psi\nabla\psi_l
= \sum^{n_F}_{l=1} K^F_{il}, \\
\widetilde{\widetilde{K}} := \int_\Omega k\nabla\Psi\nabla\Psi 
= \sum^{n_F}_{l_1=1}\sum^{n_F}_{l_2=1} \int_\Omega k\nabla\psi_{l_1}\nabla\psi_{l_2},\\
f_i := \int_\Omega f\varphi_i, \quad h_i := \int_{\Gamma_N} h\varphi_i,
\quad \widetilde{f} := \int_\Omega f\Psi, 
\quad \widetilde{h} :=\int_{\Gamma_N} h\Psi.
\end{gathered}
\end{equation*}
%
Then the linear system is
%
\begin{align*}
&K_{ij} v_j + \widetilde{K}_i c + f_i + h_i + K_{ij} g_j + K^F_{il} p_l = 0, \\
&\widetilde{K}_j v_j + \widetilde{\widetilde{K}} c + \widetilde{f} + \widetilde{h} + \widetilde{K}_j g_j + \widetilde{K}^F_l p_l = 0, \\
\end{align*}
%
or
\begin{equation*}
%
\left( \begin{array}{cc}
{\bf K} & {\bf \widetilde{K}} \\ {\bf K^T} & {\bf \widetilde{\widetilde{K}}}
\end{array} \right)
\left( \begin{array}{c}  {\bf v} \\ c  \end{array} \right)
+
\left( \begin{array}{c}  {\bf f} \\ {\bf \widetilde{f}}  \end{array} \right)
+
\left( \begin{array}{c}  {\bf h} \\ {\bf \widetilde{h}}  \end{array} \right)
+
\left( \begin{array}{c}
{\bf K^F} \\ ({\bf \widetilde{K}^F})^T
\end{array} \right)
\left( {\bf p} \right)
+
\left( \begin{array}{c}  {\bf K} \\ {\bf \widetilde{K}}^T  \end{array} \right)
\left( {\bf g} \right)
= 0,
%
\end{equation*}
%
or
%
\begin{equation*}
%
\left( \begin{array}{cc}
{\bf K} & {\bf \widetilde{K}} \\ {\bf K^T} & {\bf \widetilde{\widetilde{K}}}
\end{array} \right)
\left( \begin{array}{c}  {\bf v} \\ c  \end{array} \right)
+
\left( \begin{array}{c}  {\bf f} \\ {\bf \widetilde{f}}  \end{array} \right)
+
\left( \begin{array}{c}  {\bf h} \\ {\bf \widetilde{h}}  \end{array} \right)
+
\left( \begin{array}{cc}
{\bf K} & {\bf K^F} \\ {\bf \widetilde{K}}^T & ({\bf \widetilde{K}^F})^T
\end{array} \right)
\left( \begin{array}{c}  {\bf p} \\ {\bf g}  \end{array} \right)
= 0.
%
\end{equation*}

\end{document}
